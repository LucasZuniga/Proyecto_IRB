\documentclass[spanish, letterpaper, journal]{IEEEtran}
\usepackage{amsmath,amsfonts}
\usepackage{graphicx}
\usepackage{cite}
\usepackage{url}
\usepackage[spanish]{babel}
\usepackage[utf8]{inputenc}
\usepackage[T1]{fontenc}
\begin{document}

\title{Diseño y Desarrollo de Robots Educativos para Introducción a la Robótica mediante Fútbol de Robots}

\author{Allende, S., Lorca, S., Reyes, B., Román, C., Zúñiga, L.}

\maketitle

\begin{abstract}
    Este trabajo presenta el diseño y desarrollo de robots educativos orientados a la enseñanza introductoria de robótica mediante la implementación de fútbol de robots como plataforma experimental. El proyecto abarca desde la definición de las dimensiones físicas del robot hasta la construcción del circuito de control, la estructura mecánica y la programación inicial. Se propone un robot compacto (10×10 cm) que utiliza tres motores de corriente continua para su desplazamiento y control del balón, además de un solenoide para efectuar el disparo. Como innovación principal, se incorpora un rodillo de control de balón, el cual mejora significativamente la precisión y maniobrabilidad en comparación con la antigua pinza utilizada en versiones previas.
\end{abstract}

\section{Introducción}
\IEEEPARstart{E}{l} presente trabajo tiene como objetivo desarrollar una plataforma educativa que facilite la enseñanza de conceptos básicos de robótica mediante robots futbolistas. Este nuevo diseño busca reemplazar la versión utilizada previamente en el curso IRB2001, incorporando mejoras en el control del balón, permitiendo la participación simultánea de múltiples robots por equipo y habilitando nuevos métodos de comunicación entre el computador del estudiante y el robot.

\subsection{Objetivos del proyecto}
\begin{itemize}
    \item Diseñar un robot funcional dentro de un área máxima de 10x10 cm.
    \item Desarrollar un sistema de control capaz de manipular un balón con precisión mediante un rodillo.
    \item Implementar una estructura robusta y ligera para facilitar el movimiento.
    \item Introducir conceptos de control, sensado, comunicación y visión computacional en un entorno educativo.
\end{itemize}

\section{Análisis Inicial y Restricciones de Diseño}
Antes de iniciar el desarrollo, se evaluaron las dimensiones del laboratorio y las limitaciones de los materiales disponibles. Se determinó que una base de 10x10 cm ofrecía un equilibrio adecuado entre estabilidad, maniobrabilidad y espacio para los componentes electrónicos.

\subsection{Requerimientos de diseño}
\begin{itemize}
    \item Tamaño máximo: 10x10 cm.
    \item Peso objetivo: menos de 3 kg.
    \item Componentes principales: Raspberry Pi Pico 2 W, 2 motores DC N20, 1 solenoide, 1 rodillo de control de balón.
    \item Fuente de energía: batería LiPo de 4 celdas (14.8 V).
    \item Comunicación: WiFi (preferente) y Bluetooth para interacción con múltiples dispositivos.
    \item Capacidades funcionales:
          \begin{itemize}
              \item Mantener la posesión del balón mediante un rodillo de control.
              \item Realizar el golpeo del balón mediante un solenoide.
              \item Desplazarse de manera estable sobre la cancha.
              \item Permitir control simultáneo desde múltiples dispositivos.
          \end{itemize}
\end{itemize}

\section{Diseño Mecánico}
El diseño mecánico busca un equilibrio entre simplicidad, resistencia y precisión en el control del balón.

\subsection{Estructura y materiales}
El chasis se fabrica en materiales livianos (por ejemplo, acrílico o PLA impreso en 3D), con soportes modulares para los motores, el rodillo y la electrónica. La disposición de los componentes se optimiza para mantener un centro de masa bajo y facilitar el mantenimiento.

\subsection{Sistema de locomoción}
El robot consta de dos ruedas motorizadas junto con un apoyo tipo bola, el cual proporciona un tercer punto de apoyo sin interferir en el movimiento. Las ruedas motorizadas se encuentran en la parte delantera del robot, una a cada lado. Los motores utilizados son motores DC N20 con una caja reductora de relación 1:100.

Esta disposición permite realizar movimientos lineales y giros sobre su eje mediante control diferencial. La combinación de ambos movimientos otorga al robot desplazamientos ágiles a lo largo de toda la cancha, manteniendo el control del balón durante la maniobra.

\subsection{Sistema de control de balón}
Se reemplaza la pinza mecánica del robot utilizado anteriormente por un rodillo frontal activo que permite atrapar, retener y mantener la posesión del balón de manera continua. El rodillo gira constantemente, atrayendo el balón hacia el robot y manteniéndolo en contacto, lo que facilita el control dinámico durante el desplazamiento.

La implementación de un mecanismo de dribbling activo permite aplicar efecto de retroceso (backspin) al balón, generando un efecto de ‘‘succión’’ que mejora la retención durante aceleraciones y giros, una técnica ampliamente validada en la Small Size League de la Robocup~\cite{dribble}. Esta solución presenta ventajas significativas sobre la pinza anterior: permite mantener posesión del balón mientras el robot se desplaza, aumenta la maniobrabilidad en espacios reducidos y proporciona mayor estabilidad durante movimientos complejos. El diseño del rodillo asegura que el balón permanezca bajo control incluso durante giros cerrados o cambios bruscos de dirección.

\subsection{Mecanismo de disparo}
El disparo del balón se realiza mediante un solenoide lineal que proporciona una fuerza de impacto controlada y repetible. Al activarse, el solenoide impulsa un émbolo que golpea el balón con suficiente fuerza para enviarlo a lo largo de la cancha.

El diseño incluye amortiguación para proteger los componentes electrónicos de vibraciones y prolongar la vida útil del actuador. La temporización del disparo se controla mediante el microcontrolador, permitiendo ajustar la intensidad del golpeo según la estrategia de juego.

\section{Diseño Electrónico}
El sistema electrónico se basa en un microcontrolador Raspberry Pi Pico 2 W, elegido por su bajo costo, capacidad de procesamiento dual-core, conectividad inalámbrica integrada y facilidad de programación. La arquitectura electrónica se diseñó considerando eficiencia energética, modularidad y robustez ante condiciones de operación exigentes.

\subsection{Arquitectura del Sistema}
La arquitectura electrónica del robot se organiza en cuatro subsistemas principales que interactúan de manera coordinada: la unidad de procesamiento central (MCU), la red de distribución de energía, la interfaz de control de motores dual y el circuito de actuación del solenoide. La Fig.~\ref{fig_block_diagram} presenta un diagrama de bloques de alto nivel que ilustra el flujo de datos y potencia entre estos subsistemas.

\begin{figure}[!ht]
    \centering
    \includegraphics[width=\linewidth]{figs/block_diagram.pdf}
    \caption{Diagrama de bloques de la arquitectura electrónica. Las líneas rojas indican flujo de potencia y las azules representan flujo de datos y señales de control.}
    \label{fig_block_diagram}
\end{figure}

Como se observa en el diagrama, el sistema recibe alimentación desde una batería LiPo de 7.4V que se distribuye a través de una red regulada hacia los diferentes subsistemas. El MCU coordina todas las operaciones, procesando comandos inalámbricos (WiFi/Bluetooth), ejecutando algoritmos de control y generando señales de actuación. Los motores DC reciben señales PWM moduladas a través del driver TB6612FNG, mientras que el solenoide se activa mediante un circuito de conmutación basado en un transistor TIP102. La retroalimentación proveniente de encoders magnéticos cierra el lazo de control, permitiendo un seguimiento preciso de las velocidades de las ruedas.

\subsection{Unidad de Procesamiento Central (MCU)}
El Raspberry Pi Pico 2 W constituye el cerebro del sistema, actuando como coordinador central de todas las operaciones del robot. La selección de este microcontrolador se fundamenta en diversos criterios técnicos y pedagógicos que lo hacen adecuado para una plataforma educativa de robótica.

\textbf{Justificación de la selección:}

\textit{Facilidad de programación y versatilidad de lenguajes:}  
Una de las principales ventajas del Raspberry Pi Pico 2 W es su compatibilidad con múltiples entornos de programación, específicamente MicroPython y C/C++. Esta característica resulta fundamental para el curso donde será implementado, ya que permite:
\begin{itemize}
    \item \textit{MicroPython}: ofrece una curva de aprendizaje suave para estudiantes sin experiencia en microcontroladores. Su sintaxis intuitiva y el entorno interactivo REPL facilitan la experimentación y depuración rápida.
    \item \textit{C/C++}: proporciona acceso a bajo nivel del hardware para estudiantes avanzados que requieren optimización o funcionalidades específicas, permitiendo introducir conceptos de sistemas embebidos y gestión eficiente de recursos.
\end{itemize}

Esta dualidad permite adaptar el nivel de complejidad según el progreso del estudiante sin necesidad de cambiar de plataforma.

\textit{Arquitectura dual-core para procesamiento paralelo:}  
El procesador dual-core ARM Cortex-M33 a 133 MHz ofrece ventajas significativas. Como se detalla en la sección de software, esta característica permite:
\begin{itemize}
    \item Dedicación de un núcleo (Core 0) a tareas de \textbf{comunicación inalámbrica}, gestionando recepción y envío de paquetes UDP sin interferir en otras operaciones.
    \item Uso del segundo núcleo (Core 1) para el \textbf{control en tiempo real} mediante algoritmos PID, lectura de sensores y generación de señales de actuación.
    \item Garantizar determinismo temporal en el lazo de control, fundamental para la estabilidad del sistema.
\end{itemize}

Esta separación evita conflictos de recursos y asegura que la comunicación no introduzca latencias variables.

\textit{Capacidades GPIO flexibles:}  
El microcontrolador ofrece 26 pines GPIO programables con soporte para:
\begin{itemize}
    \item \textit{PWM}: 16 canales independientes para control de motores y modulación del solenoide.
    \item \textit{GPIO digital}: señales de dirección, activación y lectura de encoders.
    \item \textit{I2C, SPI, UART}: expansión futura con sensores adicionales.
    \item \textit{ADC}: monitoreo de voltaje de batería u otras señales analógicas.
\end{itemize}

\textbf{Interfaz con controladores de motores:}
El MCU se comunica con el TB6612FNG mediante una interfaz digital-PWM:
\begin{itemize}
    \item Señales PWM independientes para cada motor (PWM\_A y PWM\_B), típicamente entre 20–40 kHz.
    \item Señales digitales IN1/IN2 e IN3/IN4 para dirección y frenado.
    \item Señal STBY para control de estado activo/standby.
\end{itemize}

\textbf{Especificaciones técnicas principales:}
\begin{itemize}
    \item Procesador dual-core ARM Cortex-M33 a 133 MHz.
    \item 264 KB de SRAM y 2 MB de flash.
    \item WiFi 802.11n y Bluetooth 5.2 integrados.
    \item 26 pines GPIO multifunción.
    \item Consumo menor a 1 mA en modo dormido.
    \item Rango de operación 1.8V–5.5V.
    \item Interfaz USB 1.1 para programación y depuración.
\end{itemize}

\subsection{Circuitos de Accionamiento y Control}

\textbf{Driver de Motores (TB6612FNG):}
\begin{itemize}
    \item Corriente continua: 1.2A por canal (3.2A pico).
    \item Control de dirección mediante señales digitales IN1, IN2.
    \item Control de velocidad mediante PWM.
    \item Modo de frenado activo.
    \item Protección térmica y contra sobrecorriente.
\end{itemize}

\textbf{Circuito de Actuación del Solenoide (TIP102):}
\begin{itemize}
    \item Corriente soportada: hasta 8A continua.
    \item Tensión soportada: hasta 100V.
    \item Alta ganancia de corriente ($h_{FE} > 1000$).
    \item Diodo de protección en paralelo para picos inductivos.
\end{itemize}

\subsection{Gestión de Energía}

\textbf{Fuente de Alimentación:}  
Batería LiPo de 4 celdas (14.8V, 16.8V cargada) y 1500 mAh.

\textbf{Red de Distribución de Voltajes:}
\begin{itemize}
    \item 12V directo para el solenoide.
    \item 5V regulado para el TB6612FNG y motores.
    \item 3.3V regulado para el Raspberry Pi Pico 2 W.
\end{itemize}

\textbf{Protecciones:}
\begin{itemize}
    \item Diodo anti inversión.
    \item Diodo flyback en el solenoide.
    \item Switch mecánico para activación manual.
\end{itemize}

\section{Desarrollo de Software}

\subsection{Pruebas realizadas}
[COMPLETAR]

\subsection{Limitaciones observadas}
[COMPLETAR]

\section{Conclusiones y Trabajo Futuro}
El diseño presentado demuestra que es posible construir una plataforma educativa robusta, modular y escalable. La sustitución de la pinza por un rodillo mejoró considerablemente la manipulación del balón. Como trabajo futuro, se planea incorporar algoritmos de visión artificial, control autónomo y estrategias cooperativas entre múltiples robots.

\section*{Disponibilidad de Material}
Todo el material relacionado con este proyecto, incluyendo diseños mecánicos (archivos CAD), esquemáticos electrónicos (PCB y circuitos), código fuente del software (MicroPython/C) y documentación técnica, se encuentra disponible de forma abierta en el repositorio GitHub del proyecto: \url{https://github.com/LucasZuniga/Proyecto_IRB}

\section*{Agradecimientos}
Se agradece a [COMPLETAR] por facilitar el espacio y los materiales para el desarrollo del proyecto.

\bibliographystyle{IEEEtran}
\bibliography{references}

\end{document}
